%%%%%%%%%%%%%%%%%
% TEMPLATE HORIZON 2020 DATA MANAGEMENT PLAN (DMP)

% Based on the H2020 version. Available at: https://ec.europa.eu/research/participants/docs/h2020-funding-guide/cross-cutting-issues/open-access-data-management/data-management_en.htm/

% INTRODUCTION
% This Horizon 2020 DMP template has been designed to be applicable to any Horizon 2020 project that produces, collects or processes research data. You should develop a single DMP for your project to cover its overall approach. However, where there are specific issues for individual datasets (e.g. regarding openness), you should clearly spell this out.
%Guidelines on FAIR Data Management in Horizon 2020 are available in the Online Manual.

% FAIR DATA MANAGEMENT
% In general terms, your research data should be 'FAIR', that is findable, accessible, interoperable and re-usable. These principles precede implementation choices and do not necessarily suggest any specific technology, standard, or implementation-solution.
% This template is not intended as a strict technical implementation of the FAIR principles, it is rather inspired by FAIR as a general concept.
% More information about FAIR:
% FAIR data principles (FORCE11 discussion forum)
% FAIR principles (article in Nature)

% STRUCTURE OF THE TEMPLATE
% The template is a set of questions that you should answer with a level of detail appropriate to the project.
% It is not required to provide detailed answers to all the questions in the first version of the DMP that needs to be submitted by month 6 of the project. Rather, the DMP is intended to be a living document in which information can be made available on a finer level of granularity through updates as the implementation of the project progresses and when significant changes occur. Therefore, DMPs should have a clear version number and include a timetable for updates. As a minimum, the DMP should be updated in the context of the periodic evaluation/assessment of the project. If there are no other periodic reviews envisaged within the grant agreement, an update needs to be made in time for the final review at the latest.
% In the following the main sections to be covered by the DMP are outlined. At the end of the document, Table 1 contains a summary of these elements in bullet form.

% This template itself may be updated as the policy evolves.
%%%%%%%%%%%%%%%%

\documentclass{article}
\usepackage[utf8]{inputenc}
\title{Data management plan}
\author{}
\date{}

\begin{document}

\begin{titlepage}
\maketitle
\bigskip
\section*{Project  Number:}
%%%% insert project reference number
\section*{Project Acronym:}
%%%% insert acronym
\section*{Project title:}
%%%% insert project title
\bigskip
\section*{History of changes}
\begin{tabular}{ l | c | r }
   \textbf{Version} & \textbf{Publication date} & \textbf{Change} \\
   1.0 & 2020/10/09 & Initial version \\
\end{tabular}
\end{titlepage}

\newpage
\tableofcontents
\newpage

\section{Data Summary}
%%%% The questions are a guideline and should not be included in the final document. Depending on the project, not all questions need to be answered.
\begin{itemize}
\item What is the purpose of the data collection/generation and its relation to the objectives of the project?
\item What types and formats of data will the project generate/collect?
\item Will you re-use any existing data and how?
\item What is the origin of the data?
\item What is the expected size of the data?
\item To whom might it be useful ('data utility')?
\end{itemize}

\section{FAIR data}
\subsection{Making data findable, including provisions for metadata}
%%%% The questions are a guideline and should not be included in the final document. Depending on the project, not all questions need to be answered.
\begin{itemize}
    \item Are the data produced and/or used in the project discoverable with metadata, identifiable and locatable by means of a standard identification mechanism (e.g. persistent and unique identifiers such as Digital Object Identifiers)?
    \item What naming conventions do you follow?
    \item Will search keywords be provided that optimize possibilities for re-use?
    \item Do you provide clear version numbers?
    \item What metadata will be created? In case metadata standards do not exist in your discipline, please outline what type of metadata will be created and how.
\end{itemize}

\subsection{Making data openly accessible}
%%%% The questions are a guideline and should not be included in the final document. Depending on the project, not all questions need to be answered.
\begin{itemize}
    \item Which data produced and/or used in the project will be made openly available as the default? If certain datasets cannot be shared (or need to be shared under restrictions), explain why, clearly separating legal and contractual reasons from voluntary restrictions.
    \item Which data produced and/or used in the project will be made openly available as the default? If certain datasets cannot be shared (or need to be shared under restrictions), explain why, clearly separating legal and contractual reasons from voluntary restrictions.
    \item Note that in multi-beneficiary projects it is also possible for specific beneficiaries to keep their data closed if relevant provisions are made in the consortium agreement and are in line with the reasons for opting out.
    \item How will the data be made accessible (e.g. by deposition in a repository)?
    \item What methods or software tools are needed to access the data?
    \item Is documentation about the software needed to access the data included?
    \item Is it possible to include the relevant software (e.g. in open source code)?
    \item Where will the data and associated metadata, documentation and code be deposited? Preference should be given to certified repositories which support open access where possible.
    \item Have you explored appropriate arrangements with the identified repository?
    \item If there are restrictions on use, how will access be provided?
    \item Is there a need for a data access committee?
    \item Are there well described conditions for access (i.e. a machine readable license)?
    \item How will the identity of the person accessing the data be ascertained?
\end{itemize}

\subsection{Making data interoperable}
%%%% The questions are a guideline and should not be included in the final document. Depending on the project, not all questions need to be answered.
\begin{itemize}
    \item Are the data produced in the project interoperable, that is allowing data exchange and re-use between researchers, institutions, organisations, countries, etc. (i.e. adhering to standards for formats, as much as possible compliant with available (open) software applications, and in particular facilitating re-combinations with different datasets from different origins)?
    \item What data and metadata vocabularies, standards or methodologies will you follow to make your data interoperable?
    \item Will you be using standard vocabularies for all data types present in your data set, to allow inter-disciplinary interoperability?
    \item In case it is unavoidable that you use uncommon or generate project specific ontologies or vocabularies, will you provide mappings to more commonly used ontologies?
\end{itemize}

\subsection{Increase data re-use (through clarifying licences)}
%%%% The questions are a guideline and should not be included in the final document. Depending on the project, not all questions need to be answered.
\begin{itemize}
    \item How will the data be licensed to permit the widest re-use possible?
    \item When will the data be made available for re-use? If an embargo is sought to give time to publish or seek patents, specify why and how long this will apply, bearing in mind that research data should be made available as soon as possible.
    \item Are the data produced and/or used in the project useable by third parties, in particular after the end of the project? If the re-use of some data is restricted, explain why.
    \item How long is it intended that the data remains re-usable?
    \item Are data quality assurance processes described?
    \item Further to the FAIR principles, DMPs should also address:
\end{itemize}

\section{Allocation of resources}
%%%% The questions are a guideline and should not be included in the final document. Depending on the project, not all questions need to be answered.
\begin{itemize}
    \item  What are the costs for making data FAIR in your project?
    \item How will these be covered? Note that costs related to open access to research data are eligible as part of the Horizon 2020 grant (if compliant with the Grant Agreement conditions).
    \item Who will be responsible for data management in your project?
    \item Are the resources for long term preservation discussed (costs and potential value, who decides and how what data will be kept and for how long)?
\end{itemize}

\section{Data security}
%%%% The questions are a guideline and should not be included in the final document. Depending on the project, not all questions need to be answered.
\begin{itemize}
    \item What provisions are in place for data security (including data recovery as well as secure storage and transfer of sensitive data)?
    \item Is the data safely stored in certified repositories for long term preservation and curation?
\end{itemize}

\section{Ethical aspects}
%%%% The questions are a guideline and should not be included in the final document. Depending on the project, not all questions need to be answered.
\begin{itemize}
    \item Are there any ethical or legal issues that can have an impact on data sharing? These can also be discussed in the context of the ethics review. If relevant, include references to ethics deliverables and ethics chapter in the Description of the Action (DoA).
    \item Is informed consent for data sharing and long term preservation included in questionnaires dealing with personal data?
\end{itemize}

\section{Other issues}
%%%% The questions are a guideline and should not be included in the final document. Depending on the project, not all questions need to be answered.
\begin{itemize}
    \item Do you make use of other national/funder/sectorial/departmental procedures for data management? If yes, which ones?
\end{itemize}

%%%%%%%%%%%%%%%%%
%%%% FURTHER SUPPORT IN DEVELOPING YOUR DMP
%%%% The questions are a guideline and should not be included in the final document. Depending on the project, not all questions need to be answered.
%%%% The Research Data Alliance provides a Metadata Standards Directory (http://rd-alliance.github.io/metadata-directory/) that can be searched for discipline-specific standards and associated tools.
%%%% The EUDAT B2SHARE (https://b2share.eudat.eu/) tool includes a built-in license wizard that facilitates the selection of an adequate license for research data.
%%%% Useful listings of repositories include:
%%%% Registry of Research Data Repositories : https://www.re3data.org/
%%%% Some repositories like Zenodo, an OpenAIRE and CERN collaboration), allow researchers to deposit both publications and data, while providing tools to link them.
%%%% Other useful tools include DMP online (https://dmponline.dcc.ac.uk/) and platforms for making individual scientific observations available such as ScienceMatters.
%%%%%%%%%%%%%%%%%

\end{document}
